\documentclass{article}\usepackage[]{graphicx}\usepackage[]{color}
%% maxwidth is the original width if it is less than linewidth
%% otherwise use linewidth (to make sure the graphics do not exceed the margin)
\makeatletter
\def\maxwidth{ %
  \ifdim\Gin@nat@width>\linewidth
    \linewidth
  \else
    \Gin@nat@width
  \fi
}
\makeatother

\definecolor{fgcolor}{rgb}{0.345, 0.345, 0.345}
\newcommand{\hlnum}[1]{\textcolor[rgb]{0.686,0.059,0.569}{#1}}%
\newcommand{\hlstr}[1]{\textcolor[rgb]{0.192,0.494,0.8}{#1}}%
\newcommand{\hlcom}[1]{\textcolor[rgb]{0.678,0.584,0.686}{\textit{#1}}}%
\newcommand{\hlopt}[1]{\textcolor[rgb]{0,0,0}{#1}}%
\newcommand{\hlstd}[1]{\textcolor[rgb]{0.345,0.345,0.345}{#1}}%
\newcommand{\hlkwa}[1]{\textcolor[rgb]{0.161,0.373,0.58}{\textbf{#1}}}%
\newcommand{\hlkwb}[1]{\textcolor[rgb]{0.69,0.353,0.396}{#1}}%
\newcommand{\hlkwc}[1]{\textcolor[rgb]{0.333,0.667,0.333}{#1}}%
\newcommand{\hlkwd}[1]{\textcolor[rgb]{0.737,0.353,0.396}{\textbf{#1}}}%

\usepackage{framed}
\makeatletter
\newenvironment{kframe}{%
 \def\at@end@of@kframe{}%
 \ifinner\ifhmode%
  \def\at@end@of@kframe{\end{minipage}}%
  \begin{minipage}{\columnwidth}%
 \fi\fi%
 \def\FrameCommand##1{\hskip\@totalleftmargin \hskip-\fboxsep
 \colorbox{shadecolor}{##1}\hskip-\fboxsep
     % There is no \\@totalrightmargin, so:
     \hskip-\linewidth \hskip-\@totalleftmargin \hskip\columnwidth}%
 \MakeFramed {\advance\hsize-\width
   \@totalleftmargin\z@ \linewidth\hsize
   \@setminipage}}%
 {\par\unskip\endMakeFramed%
 \at@end@of@kframe}
\makeatother

\definecolor{shadecolor}{rgb}{.97, .97, .97}
\definecolor{messagecolor}{rgb}{0, 0, 0}
\definecolor{warningcolor}{rgb}{1, 0, 1}
\definecolor{errorcolor}{rgb}{1, 0, 0}
\newenvironment{knitrout}{}{} % an empty environment to be redefined in TeX

\usepackage{alltt}
\usepackage{amsmath}
\usepackage{graphicx,psfrag,epsf}
\usepackage{enumerate}

\addtolength{\oddsidemargin}{-.7in}%
\addtolength{\evensidemargin}{-.5in}%
\addtolength{\textwidth}{1in}%
\addtolength{\textheight}{1.3in}%
\addtolength{\topmargin}{-.8in}%

\newcommand{\bm}{\mathbf}
\newcommand{\bs}{\boldsymbol}
\IfFileExists{upquote.sty}{\usepackage{upquote}}{}
\begin{document}
\def\spacingset#1{\renewcommand{\baselinestretch}%
{#1}\small\normalsize} \spacingset{1}

Let $\bm{D}_j = \bs\Phi_j^C$ and consider the adjustment matrices 
\[
\bm{A}_j = \bm{D}_j' \bm{B}_j^{+/2} \bm{D}_j, \]
where $\bm{B}_j = \bm{D}_j \left(\bm{I} - \bm{H}\right)_j \bs\Phi \left(\bm{I} - \bm{H}\right)_j' \bm{D}_j'$ and $\bm{B}_j^{+/2}$ denotes the symmetric square root of the Moore-Penrose inverse of $\bm{B}_j$. Then in order for $\bm{V}^R$ to be exactly model-unbiased, we must have that 
\[
\bm{X}_j' \bm{W}_j \bm{D}_j' \bm{B}_j^{+/2} \bm{B}_j \bm{B}_j^{+/2} \bm{D}_j \bm{W}_j \bm{X}_j = \bm{X}_j' \bm{W}_j \bm{D}_j' \bm{D}_j \bm{W}_j \bm{X}_j \]
or equivalently that 
\[
\bm{X}_j' \bm{W}_j \bm{D}_j' \bm{B}_j^+ \bm{B}_j \bm{D}_j \bm{W}_j \bm{X}_j = \bm{X}_j' \bm{W}_j \bm{D}_j' \bm{D}_j \bm{W}_j \bm{X}_j, \]
where $\bm{B}_j^+$ is the Moore-Penrose inverse of $\bm{B}_j$. 

Now consider the rank-decomposition of $\left(\bm{I} - \bm{H}\right)_j = \bm{C}\bm{R}$ for $n_j \times r$ matrix $\bm{C}$ with full column-rank and $\bm{R}$ is $r \times N$ with full row-rank. Then it can be verified that \[
\bm{B}_j^+ = \bm{D}_j \bm{C}\left(\bm{C}'\bm{D}_j'\bm{D}_j \bm{C}\right)^{-1} \left(\bm{R}\bs\Phi\bm{R}'\right)^{-1} \left(\bm{C}'\bm{D}_j'\bm{D}_j \bm{C}\right)^{-1} \bm{C}' \bm{D}_j'
\]
and therefore that \[
\bm{B}_j^+ \bm{B}_j = \bm{D}_j \bm{C}\left(\bm{C}'\bm{D}_j'\bm{D}_j \bm{C}\right)^{-1} \bm{C}' \bm{D}_j'.
\]
Thus, the question is to identify conditions on $\bm{X}_j$ under which the following equality holds: \[
\bm{X}_j' \bm{W}_j \bm{D}_j' \bm{D}_j \bm{C}\left(\bm{C}'\bm{D}_j'\bm{D}_j \bm{C}\right)^{-1} \bm{C}' \bm{D}_j' \bm{D}_j \bm{W}_j \bm{X}_j = \bm{X}_j' \bm{W}_j \bm{D}_j' \bm{D}_j \bm{W}_j \bm{X}_j. \]
It seems like to answer this question, we need to find an explicit expression for $\bm{C}$ in terms of the components of $\left(\bm{I} - \bm{H}\right)_j$. Not sure how to do that....


\end{document}
