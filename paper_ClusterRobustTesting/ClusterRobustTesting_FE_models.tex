\documentclass[12pt]{article}\usepackage[]{graphicx}\usepackage[]{color}
%% maxwidth is the original width if it is less than linewidth
%% otherwise use linewidth (to make sure the graphics do not exceed the margin)
\makeatletter
\def\maxwidth{ %
  \ifdim\Gin@nat@width>\linewidth
    \linewidth
  \else
    \Gin@nat@width
  \fi
}
\makeatother

\definecolor{fgcolor}{rgb}{0.345, 0.345, 0.345}
\newcommand{\hlnum}[1]{\textcolor[rgb]{0.686,0.059,0.569}{#1}}%
\newcommand{\hlstr}[1]{\textcolor[rgb]{0.192,0.494,0.8}{#1}}%
\newcommand{\hlcom}[1]{\textcolor[rgb]{0.678,0.584,0.686}{\textit{#1}}}%
\newcommand{\hlopt}[1]{\textcolor[rgb]{0,0,0}{#1}}%
\newcommand{\hlstd}[1]{\textcolor[rgb]{0.345,0.345,0.345}{#1}}%
\newcommand{\hlkwa}[1]{\textcolor[rgb]{0.161,0.373,0.58}{\textbf{#1}}}%
\newcommand{\hlkwb}[1]{\textcolor[rgb]{0.69,0.353,0.396}{#1}}%
\newcommand{\hlkwc}[1]{\textcolor[rgb]{0.333,0.667,0.333}{#1}}%
\newcommand{\hlkwd}[1]{\textcolor[rgb]{0.737,0.353,0.396}{\textbf{#1}}}%

\usepackage{framed}
\makeatletter
\newenvironment{kframe}{%
 \def\at@end@of@kframe{}%
 \ifinner\ifhmode%
  \def\at@end@of@kframe{\end{minipage}}%
  \begin{minipage}{\columnwidth}%
 \fi\fi%
 \def\FrameCommand##1{\hskip\@totalleftmargin \hskip-\fboxsep
 \colorbox{shadecolor}{##1}\hskip-\fboxsep
     % There is no \\@totalrightmargin, so:
     \hskip-\linewidth \hskip-\@totalleftmargin \hskip\columnwidth}%
 \MakeFramed {\advance\hsize-\width
   \@totalleftmargin\z@ \linewidth\hsize
   \@setminipage}}%
 {\par\unskip\endMakeFramed%
 \at@end@of@kframe}
\makeatother

\definecolor{shadecolor}{rgb}{.97, .97, .97}
\definecolor{messagecolor}{rgb}{0, 0, 0}
\definecolor{warningcolor}{rgb}{1, 0, 1}
\definecolor{errorcolor}{rgb}{1, 0, 0}
\newenvironment{knitrout}{}{} % an empty environment to be redefined in TeX

\usepackage{alltt}
\usepackage{amsmath}
\usepackage{amsthm}
\usepackage{graphicx,psfrag,epsf}
\usepackage{enumerate}
%\usepackage{natbib}
\RequirePackage[natbibapa]{apacite}

%\pdfminorversion=4
% NOTE: To produce blinded version, replace "0" with "1" below.
\newcommand{\blind}{0}

% DON'T change margins - should be 1 inch all around.
\addtolength{\oddsidemargin}{-.7in}%
\addtolength{\evensidemargin}{-.5in}%
\addtolength{\textwidth}{1in}%
\addtolength{\textheight}{1.3in}%
\addtolength{\topmargin}{-.8in}%

\usepackage[textwidth=1in, textsize=tiny]{todonotes}
%\usepackage[disable]{todonotes}

\newtheorem{thm}{Theorem}
\newtheorem{lem}{Lemma}

\newcommand{\Prob}{\text{Pr}}
\newcommand{\E}{\text{E}}
\newcommand{\Cov}{\text{Cov}}
\newcommand{\corr}{\text{corr}}
\newcommand{\Var}{\text{Var}}
\newcommand{\iid}{\stackrel{\text{iid}}{\sim}}
\newcommand{\tr}{\text{tr}}
\newcommand{\bm}{\mathbf}
\newcommand{\bs}{\boldsymbol}
\IfFileExists{upquote.sty}{\usepackage{upquote}}{}
\begin{document}
%\SweaveOpts{concordance=TRUE}

\def\spacingset#1{\renewcommand{\baselinestretch}%
{#1}\small\normalsize} \spacingset{1}


%%%%%%%%%%%%%%%%%%%%%%%%%%%%%%%%%%%%%%%%%%%%%%%%%%%%%%%%%%%%%%%%%%%%%%%%%%%%%%

\if0\blind
{
  \title{\bf Small sample methods for cluster-robust variance estimation and hypothesis testing in fixed-effect models}
  \author{\\James E. Pustejovsky\thanks{
    The authors thank Dan Knopf for helpful discussions about the linear algebra behind the cluster-robust variance estimator. Coady Wing,...}\hspace{.2cm}\\
    Department of Educational Psychology \\ 
    University of Texas at Austin\\ \\
    and \\ \\
    Elizabeth Tipton \\
    Department of Human Development \\ 
    Teachers College, Columbia University}
  \maketitle
} \fi

\if1\blind
{
  \bigskip
  \bigskip
  \bigskip
  \begin{center}
    {\LARGE\bf Small sample methods for cluster-robust variance estimation and hypothesis testing in fixed-effect models}
\end{center}
  \medskip
} \fi

\bigskip
\begin{abstract}
The text of your abstract.  200 or fewer words.
\end{abstract}

\noindent%
{\it Keywords:}  3 to 6 keywords, that do not appear in the title
\vfill

\newpage
\spacingset{1.45} % DON'T change the spacing!

\section{INTRODUCTION}
\label{sec:intro}



Fixed-effect models are an important tool for applied economic analysis. 
Controlling for unobserved confounding factors.
Leading cases: panel models for repeated measurements on a set of individuals, organizations, or other aggregate units; block-randomized experiments (or analogous observational studies). 

Bertrand et al. highlight the need to use cluster-robust variance estimation.

Problems with standard CRVE.

Recent solutions. 

Data encountered in empirical economic research often involves clusters of dependent observations. 
For example, individuals are often clustered by countries, regions, or states; by firms, organizations, or schools; or by time-periods or follow-up waves.\todo{Are we trying to reference the panel data case here? Clustering by time period doesn't seem quite right.} 
This clustering is typically accounted for in analyses through the use of cluster-robust variance estimation (CRVE), an analog to the heteroscedasticity-robust standard errors developed by Huber (1967)\todo{Add citation}, \citet{eicker1967limit}, and \citet{White1980heteroskedasticity} to account for non-constant variance in ordinary least squares. 
The use of CRVE is widespread, as evidenced by the large number of citations to key articles in the field (e.g., 849 cites for \citet{Wooldridge2003cluster}), 
the large number of citations overall (i.e., over 11,000 for "clustered standard errors" in Google Scholar), and the large number of articles employing the methods in economics journals (i.e., over 500 citations). 

CRVE is routinely used in order to test hypotheses involving either individual coefficients or sets of multiple constraints on the regression specification. 
The theory behind CRVE is asymptotic in the number of clusters, and recently, researchers have turned attention to the performance of these tests in small and moderate samples. 
\citet{Cameron2015practitioners} provide a thorough review of this literature, including a discussion of current practice, possible solutions, and open problems. 
They highlight well known results that in small samples, conventional CRVE has a downward bias and hypothesis tests based on CRVEs can have Type-I error rates that are far from the nominal level of the test.
Moreover, they review recent research showing that the small-sample corrections for t-tests typically found in software such as Stata and SAS are inadequate. 
In the course of reviewing a variety of recent proposals for addressing these problems, Cameron and Miller highlight a potentially promising method called bias-reduced linearization method (BRL), introduced by \citet{McCaffrey2001generalizations} and \citet{Bell2002bias}. 
BRL entails correcting the downward bias of the most common CRVE so that it is exactly unbiased under a working model specified by the analyst, while also remaining asymptotically consistent under arbitrary true variance structures. 
Simulations reported \citet{Bell2002bias} demonstrate that the BRL correction serves to reduce the bias of the CRVE even when the working model is mis-specified. 
The same authors also proposed and studied small-sample corrections to t-tests based on the BRL variance estimator, based on Satterthwaite \citep{Bell2002bias} or saddlepoint approximations \citep{McCaffrey2006improved}.

Despite promising simulation evidence that BRL performs well \citep[e.g.,][]{Imbens2012robust}, several problems arise in implementing the method in practice. 
Two of these problems arise in the analysis of panel data, where, following Bertrand et al (2004), it is considered best practice to account for clustering both as a fixed effect and a random effect (through CRVE) in analyses. 
One approach to do so is to include dummies for each cluster are in the analysis. However, \citet{Angrist2009mostly} argue, if this approach is used, the BRL adjustment breaks down and cannot be implemented.
A second approach (which is computationally more efficient) is to instead absorb the fixed effects. As \citet{Cameron2015practitioners} highlight, however, this absorption approach can lead to sometimes substantially different standard errors. 
Finally, while \citet{Bell2002bias} provide a method for conducting single parameter tests, no such small-sample method has been provided for multiparameter tests.
These tests occur commonly in the broader economics literature and are found not only in panel data (e.g., the Hausman test), but also more broadly in seemingly unrelated regression models, and when analyzing experimental data (e.g., baseline equivalence), particularly when there are multiple treatment groups. 

In this paper, we address each of these three concerns, in the end articulating a BRL methodology that is suitable for everyday econometric practice. 
To do so, we begin by reviewing the the small sample CRVE method that is standard in practice in most software applications. 
In the next section of the paper, we review the BRL correction to the CRVE estimator, and present advances that address the first two concerns above. 
First, we demonstrate that using generalized inverses to calculate the BRL adjustment matrices address the rank-deficiency problem that arises when including cluster fixed effects, while retaining the property that CRVEs based on the adjustment matrices are unbiased under a working model. 
Second, we describe how to apply BRL when fitting models that absorb the cluster fixed effects prior to parameter estimation. 
Here we prove that under a particular parameterization, the BRL based CRVE estimator is the same regardless of the estimation method used.
In the next section of the paper, we address the use of CRVE for hypothesis testing.
Here we propose a method of testing multiple-constraint hypotheses (i.e., F-tests) based on CRVE with the BRL adjustments, and show that the t-test proposed by \citet{Bell2002bias} is a special case. 
In support of this third aim, we provide simulation evidence that the proposed small-sample F-test offers drastic improvements over commonly implemented alternatives and performs comparably with current state-of-the-art methods such as the cluster-wild bootstrap procedure described by \citet{Cameron2008bootstrap} and \citet{Webb2013wild}. 
To date, the Wild bootstrap (and other resampling methods) are the 'best practice' with small samples, and we show that the BRL method performs just as well statistically.\todo{Does it?}
We conclude the paper with a set of three examples comparing results from these three approaches to illustrate the breadth of application, and a discussion of important considerations for practice.  
In the discussion section that follows, we then highlight why the BRL approach given here is potentially more useful in practice, and should become the standard default CRVE method used in all analyses in econometrics.

\subsection{Econometric framework}

We consider a generic fixed effects model in which
\begin{equation}
\label{eq:fixed_effects}
\bm{y}_j = \bm{R}_j \bs\beta + \bm{S}_j \bs\gamma + \bm{T}_j \bs\delta + \bs\epsilon_j,
\end{equation}
where $\bm{R}_j$ is an $n_j \times r$ matrix of covariates, $\bm{S}_j$ is an $n_j \times s$ matrix describing fixed effects that vary across clusters, and $\bm{T}_j$ is an $n_j \times t$ matrix describing fixed effects that are identified only within clusters. For example, in a balanced state-by-year panel model where the variance is estimated by clustering on states, $\bm{T}_j$ would consist of an indicator for state $j$, $\bm{S}_j$ would include indicators for each time period, and $\bm{R}_j$ would include a policy indicator or set of indicators.

We assume that $\E\left(\bs\epsilon_j\left|\bm{R}_j,\bm{S}_j, \bm{T}_j\right.\right) = \bm{0}$ and $\Var\left(\bs\epsilon_j\left|\bm{R}_j,\bm{S}_j,\bm{T}_j\right.\right) = \bs\Sigma_j$, for $j = 1,...,m$, where the form of $\bs\Sigma_1,...,\bs\Sigma_m$ may be unknown but the errors are independent across clusters. 
For notational convenience, let $\bm{U}_j = \left[\bm{R}_j \ \bm{S}_j \right]$ denote the set of predictors that vary across clusters, $\bm{X}_j = \left[\bm{U}_j \ \bm{T}_j \right]$ denote the full set of predictors, $\bs\alpha = \left(\bs\beta', \bs\gamma', \bs\delta' \right)'$, and $x = r + s + t$.
Denote the total number of individual observations by $N = \sum_{j=1}^m n_j$.
Let $\bm{y}$, $\bm{R}$, $\bm{S}$, $\bm{T}$, $\bm{U}$, and $\bm{X}$ denote the matrices obtained by stacking their corresponding components, as in $\bm{R} = \left(\bm{R}_1' \ \bm{R}_2' \ \cdots \ \bm{R}_m'\right)'$. 

In this model, inferential interest is confined to $\bs\beta$ and the fixed effects $\bs\gamma$ and $\bs\delta$ are treated as nuisance parameters. The distinction between the covariates $\bm{R}_j$ versus the fixed effects $\left[\bm{S}_j \ \bm{T}_j\right]$ thus depends on context and the analyst's inferential goals. However, the distinction between the two fixed effect matrices $\bm{S}_j$ and $\bm{T}_j$ is unambiguous, in that the within-cluster fixed effects satisfy $\bm{T}_j \bm{T}_k' = \bm{0}$ for $j \neq k$. We further assume that $\left(\bm{U}'\bm{U} - \bm{U}_j'\bm{U}_j\right)$ is of full rank for $j = 1,...,m$.

We shall consider weighted least-squares (WLS) estimation of $\bs\beta$. 
For each cluster $j$, let $\bm{W}_j$ be a symmetric, $n_j \times n_j$ weighting matrix of full rank. 
The WLS framework includes the unweighted case (where $\bm{W}_j = \bm{I}_j$, an identity matrix), as well as feasible GLS.\footnote{
The WLS estimator also encompasses the estimator proposed by \citet{Ibragimov2010tstatistic} for clustered data. 
Assuming that $\bm{X}_j$ has rank $p$ for $j = 1,...,m$, their proposed approach involves estimating $\bs\beta$ separately within each cluster and taking the simple average of these estimates. 
The resulting average is equivalent to the WLS estimator with weights $\bm{W}_j = \bm{X}_j \left(\bm{X}_j'\bm{X}_j\right)^{-2} \bm{X}_j$.} 
In the latter case, it is assumed that $\Var\left(\bm{e}_j\left|\bm{X}_j\right.\right) = \bs\Phi_j$, where $\bs\Phi_j$ is a known function of a low-dimensional parameter. 
For example, an auto-regressive error structure might be posited to describe repeated measures on an individual over time. 
The weighting matrices are then taken to be $\bm{W}_j = \hat{\bs\Phi}_j^{-1}$, where the $\hat{\bs\Phi}_j$ are constructed from estimates of the variance parameter.
Finally, for analysis of data from complex survey designs, WLS may be used with sampling weights in order to account for unequal selection probabilities.

Several approaches computing the WLS estimator are possible. 
One possibility is to calculate WLS estimates of the full parameter vector $\bs\alpha$ directly. 
However, this method can be computationally intensive and numerically inaccurate if the fixed effects specification is large (i.e., $s + t$ large). 
An alternative is to first absorb the fixed effect specification. We shall describe the latter approach because it is more efficient and numerically accurate.

Denote the full block-diagonal weighting matrix as $\bm{W} = \text{diag}\left(\bm{W}_1,...,\bm{W}_m\right)$.
Let $\bm{K}$ be the $x \times r$ matrix that selects the covariates of interest, so that $\bm{X} \bm{K} = \bm{R}$ and $\bm{K}'\bs\alpha = \bs\beta$.
For a generic matrix $\bm{Z}$ of full column rank, let $\bm{M_Z} = \left(\bm{Z}'\bm{W}\bm{Z}\right)^{-1}$ and $\bm{H_Z} = \bm{Z}\bm{M_Z}\bm{Z}'\bm{W}$. 

The absorption technique involves obtaining the residuals from the regression of $\bm{y}$ on $\bm{T}$ and from the multivariate regressions of $\bm{U} = [\bm{R}\ \bm{S}]$ on $\bm{T}$. 
The $\bm{y}$ residuals and $\bm{R}$ residuals are then regressed on the $\bm{S}$ residuals. 
Finally, these twice-regressed $\bm{y}$ residuals are regressed on the twice-regressed $\bm{R}$ residuals to obtain the WLS estimates of $\bs\beta$. 
Let $\bm{\ddot{S}} = \left(\bm{I} - \bm{H_T}\right)\bm{S}$, $\bm{\ddot{R}} = \left(\bm{I} - \bm{H_{\ddot{S}}}\right)\left(\bm{I} - \bm{H_T}\right)\bm{R}$, and $\bm{\ddot{y}} = \left(\bm{I} - \bm{H_{\ddot{S}}}\right)\left(\bm{I} - \bm{H_T}\right)\bm{y}$. 
In what follows, subscripts on $\bm{\ddot{R}}$, $\bm{\ddot{S}}$,  $\bm{\ddot{U}}$, and $\bm{\ddot{y}}$ refer to the rows of these matrices corresponding to a specific cluster. 
The WLS estimator of $\bs\beta$ can then be written as
\begin{equation}
\label{eq:WLS}
\bs{\hat\beta} = \bm{M_{\ddot{R}}} \sum_{j=1}^m \bm{\ddot{R}}_j' \bm{W}_j \bm{\ddot{y}}_j. 
\end{equation}
This estimator is algebraically identical to the direct WLS estimator based on the full set of predictors, \[
\bs{\hat\beta} = \bm{K}'\bm{M_X} \sum_{j=1}^m \bm{X}_j' \bm{W}_j \bm{y}_j,
\]
but avoids the need to solve a system of $x$ linear equations.

The variance of the WLS estimator is 
\begin{equation}
\label{eq:var_WLS}
\Var\left(\bs{\hat\beta}\right) = \bm{M_{\ddot{R}}}\left(\sum_{j=1}^m \bm{\ddot{R}}_j' \bm{W}_j \bs\Sigma_j \bm{W}_j\bm{\ddot{R}}_j\right) \bm{M_{\ddot{R}}},
\end{equation}
which depends upon the unknown variance matrices $\bm\Sigma_j$. 
One approach to estimating this variance is based on a parametric model for the error structure. 
If this approach is used, each $\bs\Sigma_j$ is substituted with an estimate $\bs{\hat\Phi}_j$, producing the model-based variance estimator
\begin{equation}
\label{eq:V_model}
\bm{V}^M = \bm{M_{\ddot{R}}}\left(\sum_{j=1}^m \bm{\ddot{R}}_j' \bm{W}_j \bs{\hat\Phi}_j \bm{W}_j\bm{\ddot{R}}_j\right) \bm{M_{\ddot{R}}}.
\end{equation}
However, if the working model is mis-specified, the model-based variance estimator will be inconsistent and inferences based upon it will be invalid.\todo{Do we even need to write out the formula for $V^M$?}

\subsection{Standard CRVE}

Cluster-robust variance estimators provide a means of estimating $\Var\left(\bs{\hat\beta}\right)$ and testing hypotheses regarding $\bs\beta$ in the absence of a valid parametric model for the error structure.\todo{Citations on originators of CRVE? Would be a long list...} 
They are thus a generalization of heteroskedasticity-consistent (HC) variance estimators.
Like the HC estimators, several different variants have been proposed, with different rationales and different finite-sample properties. 
Each of these are of the form
\begin{equation}
\label{eq:V_small}
\bm{V}^{CR} = \bm{M_{\ddot{R}}}\left(\sum_{j=1}^m \bm{\ddot{R}}_j'\bm{W}_j \bm{A}_j \bm{e}_j \bm{e}_j' \bm{A}_j' \bm{W}_j \bm{\ddot{R}}_j\right) \bm{M_{\ddot{R}}},
\end{equation}
for some $n_j$ by $n_j$ adjustment matrix $\bm{A}_j$. 
The form of these adjustments parallels those of the heteroscedastity-consistent (HC) variance estimators proposed by \citet*{MacKinnon1985some}. 
Setting $\bm{A}_j = \bm{I}_j$, an $n_j \times n_j$ identity matrix, results in the most basic form, described by \citet{Liang1986longitudinal}. 
Following \citet{Cameron2015practitioners}, we refer to this estimator as $\bm{V}^{CR0}$. 
Setting $\bm{A}_j = c\bm{I}_j$, where $c = \sqrt{(m/(m-1))(N/(N - p))}$, results in a slightly larger estimator, denoted $\bm{V}^{CR1}$.
Note that when $N >> p$, $c \approx \sqrt{m/(m-1)}$, and some software uses the latter approximation.
Both the CR0 and CR1 estimators rely on asymptotic properties of the residuals in order to consistently estimate $\bs\Sigma_j$. 
The correction constant used in the CR1 estimator does not depend on $\bm{X}_j$, and so cannot account for features of the covariates that might cause the cross-product of the residuals to better or worse estimates of the true variance. 

Several further small-sample corrections for CRVE do account for features of the covariates.
The BRL approach (described in the next section) is an extension of the HC2 estimator for regressions with heteroskedastic but uncorrelated errors; we therefore refer to it as CR2. 
A further alternative is CR3, which uses adjustment matrices given by $\bm{A}_j = \left(\bm{I} - \bm{\ddot{R}}_j \bm{M_{\ddot{R}}}\bm{\ddot{R}}_j'\bm{W}_j\right)^{-1}$. The CR3 estimator closely approximates the jackknife re-sampling variance estimator \citep{Bell2002bias, Mancl2001covariance}. 

\section{BIAS REDUCED LINEARIZATION}
\label{sec:BRL}

The BRL approach chooses adjustment matrices so that the variance estimator is exactly unbiased under a specific working model for the data. 
It is therefore directly analogous to the HC2 heteroskedasticity-robust estimator, which is exactly unbiased under homoskedasticity. 
\citet{Bell2002bias} developed the BRL estimator for linear regression models in which the errors have an unknown dependence structure within clusters. 
However, their implementation is not applicable to many fixed effect models, where the adjustment matrices may be undefined. 
For instance, \citet{Angrist2009mostly} pointed out that Bell and McCaffrey's approach cannot be applied in balanced state-by-year panels with fixed effects for states and for years because the adjustment matrices involve inverses of matrices that are not of full rank.
The form of the Bell and McCaffrey matrices also varies depending on whether fixed effects are absorbed or estimated directly by WLS, which is undesirable. Our implementation of BRL addresses both of these issues and can be implemented in models with quite general fixed effects specifications. It reduces to Bell and McCaffrey's implementation for models without fixed effects. 

Let $\bs\Phi_j$ be a working model for the covariance of the errors in cluster $j$, and denote $\bs\Phi = \text{diag}\left(\bs\Phi_1,...,\bs\Phi_m\right)$. Consider adjustment matrices satisfying the following criterion:
\begin{equation}
\label{eq:CR2_criterion}
\bm{\ddot{R}}_j' \bm{W}_j \bm{A}_j \left(\bm{I} - \bm{H_X}\right)_j \bs\Phi \left(\bm{I} - \bm{H_X}\right)_j' \bm{A}_j' \bm{W}_j \bm{\ddot{R}}_j = \bm{\ddot{R}}_j' \bm{W}_j \bs\Phi_j \bm{W}_j \bm{\ddot{R}}_j,
\end{equation}
where $\left(\bm{I} - \bm{H_X}\right)_j$ denotes the rows of $\bm{I} - \bm{H_X}$ corresponding to cluster $j$. 
A variance estimator that uses such adjustment matrices will be exactly unbiased when the working model is correctly specified.\footnote{Note that this criterion differs from the criterion used by \citet{Bell2002bias} in that it pre- and post-multiplies both sides by $\bm{W}_j\bm{\ddot{R}}_j$. 
As will be seen, this modification justifies the use of generalized matrix inverses in calculating the adjustment matrices, thus avoiding rank-deficiency problems that would otherwise leave them undefined.}
When the working model deviates from the true covariance $\bs{\Sigma}_j$, the variance estimator remains biased. However, \citet{Bell2002bias} showed that the CR2 estimator still greatly reduces the bias compared to the more basic CR0 and CR1 estimators (thus the name ``bias reduced linearization''). 
Extensive simulation results indicate that the remaining bias is typically minimal, even for large deviations from the assumed structure (CITE).
Furthermore, as the number of clusters increases, the reliance on the working model diminishes. 
In a sense, CR2 provides necessary scaffolding in the small sample case, which falls away when there is sufficient data.

Criterion (\ref{eq:CR2_criterion}) does not uniquely define $\bm{A}_j$. Following \citet{McCaffrey2001generalizations}, we propose to use a symmetric solution in which
\begin{equation}
\label{eq:CR2_adjustment}
\bm{A}_j = \bm{D}_j' \bm{B}_j^{+1/2} \bm{D}_j,
\end{equation}
where $\bm{D}_j$ is the upper-right triangular Cholesky factorization of $\hat{\bs\Phi}_j$, 
\begin{equation}
\label{eq:CR2_Bmatrix}
\bm{B}_j = \bm{D}_j\left(\bm{I} - \bm{H_{\ddot{R}}}\right)_j \left(\bm{I} - \bm{H_{\ddot{S}}}\right) \left(\bm{I} - \bm{H_T}\right) \bs\Phi \left(\bm{I} - \bm{H_T}\right)' \left(\bm{I} - \bm{H_{\ddot{S}}}\right)' \left(\bm{I} - \bm{H_{\ddot{R}}}\right)_j' \bm{D}_j',
\end{equation}
and $\bm{B}_j^{+1/2}$ is the symmetric square root of the Moore-Penrose inverse of $\bm{B}_j $. 
The Moore-Penrose inverse is well-defined and unique even when $\bm{B}_j$ is not of full rank \citep[][Thm. 9.18]{Banerjee2014linear}.
Theorem \ref{thm:BRL_FE} in Appendix \ref{app:theorems} shows that the adjustment matrices given by (\ref{eq:CR2_adjustment}) and (\ref{eq:CR2_Bmatrix}) satisfy criterion (\ref{eq:CR2_criterion}). Furthermore, because the adjustment matrices are defined in terms of all three components of the predictors ($\bm{R}$,$\bm{S}$, and $\bm{T}$), they are invariant to whether the model is estimated by direct WLS estimation or after absorbing some or all of the fixed effects. 

In many applications, it will make sense to choose weighting matrices that are the inverses of the working covariance model, so that $\bm{W}_j = \bs\Phi_j^{-1}$. In this case, the adjustment matrices can be calculated using $\bm{\tilde{B}}_j$ in place of $\bm{B}_j$, where
\begin{equation}
\label{eq:CR2_B_tilde}
\bm{\tilde{B}}_j = \bm{D}_j\left(\bm{I} - \bm{H_{\ddot{R}}}\right)_j \left(\bm{I} - \bm{H_{\ddot{S}}}\right) \bs\Phi \left(\bm{I} - \bm{H_{\ddot{S}}}\right)' \left(\bm{I} - \bm{H_{\ddot{R}}}\right)_j' \bm{D}_j'.
\end{equation}
Theorem \ref{thm:absorb} in Appendix \ref{app:theorems} demonstrates that using $\bm{\tilde{B}}_j$ rather than $\bm{B}_j$ leads to algebraically identical adjustment matrices; the form of $\bm{\tilde{B}}_j$ is simply more convenient for computation.
In the simple case of ordinary (unweighted) least squares, in which the working variance model posits that the errors are all independent and homoskedastic and $\bm{W} = \bs\Phi = \bm{I}$, the adjustment matrices simplify further to \[
\bm{A}_j = \left(\bm{I}_j - \bm{\ddot{U}}_j\left(\bm{\ddot{U}}'\bm{\ddot{U}}\right)^{-1}\bm{\ddot{U}}_j'\right)^{+1/2},\]
where $\bm{\ddot{U}} = \left(\bm{I} - \bm{H_T}\right)\bm{U}$.

The remainder of the paper considers hypothesis testing procedures based on the BRL variance estimator, $\bm{V}^{CR2}$. 


\section{HYPOTHESIS TESTING}
\label{sec:testing}

Wald-type test statistics based on CRVEs are often used to test hypotheses regarding the coefficients of the regression specification. 
Such procedures are justified based on the asymptotic behavior of robust Wald statistics as the number of clusters grows large (i.e., as $m \to \infty$). 
However, evidence from a wide variety of contexts indicates that the asymptotic limiting distribution can be a very poor approximation when the number of clusters is small, even when small-sample corrections such as CR2 are employed \citep{Bell2002bias, Bertrand2004how, Cameron2008bootstrap}. 
Furthermore, the accuracy of asymptotic approximations depends on design features such as the degree of imbalance in the covariates, skewness of the covariates, and similarity of cluster sizes \citep{McCaffrey2001generalizations, Tipton2015small, Webb2013wild}. 
Consequently, no simple rule-of-thumb exists for what constitutes an adequate sample size to trust the asymptotic test. 
This provides motivation for development of general-purpose hypothesis testing procedures that have accurate rejection rates in small samples.

We will consider linear constraints on $\bs\beta$, where the null hypothesis has the form $H_0: \bm{C}\bs\beta = \bm{d}$ for fixed $q \times r$ matrix $\bm{C}$ and $q \times 1$ vector $\bm{d}$. 
The cluster-robust Wald statistic  is then
\begin{equation}
\label{eq:Wald_stat}
Q = \left(\bm{C}\bs{\hat\beta} - \bm{d}\right)'\left(\bm{C} \bm{V}^{CR} \bm{C}'\right)^{-1}\left(\bm{C}\bs{\hat\beta} - \bm{d}\right),
\end{equation}
where $\bm{V}^{CR}$ is any of the cluster-robust estimators described above.
The asymptotically valid Wald test rejects $H_0$ at level $\alpha$ if $Q$ exceeds $\chi^2(\alpha; q)$, the $\alpha$ critical value from a chi-squared distribution with $q$ degrees of freedom. \todo{Citations to evidence that asymptotic test is way too liberal?} 
When the number of clusters is small, it is common to instead use the test statistic $F = Q/q$, compared to the $F(q, m - 1)$ reference distribution.\todo{Is this really standard?} 

Small-sample adjustments to hypothesis tests based on CRVE have largely focused on tests for single coefficients. The following subsection reviews approaches for one-dimensional hypothesis tests, with special attention to the Satterthwaite approximation approach proposed by \citet{Bell2002bias}. 
We then propose a method for testing more general, $q$-dimensional linear hypotheses regarding $\bs\beta$. Our approach is similar to a Satterthwaite approximation, in that it involves approximating the distribution of $Q$ using an $F$ distribution with estimated degrees of freedom. 

\subsection{Small-sample corrections for t-tests}

Consider testing the hypothesis $H_0: \bm{c}'\bs\beta = 0$ for some fixed $r \times 1$ contrast vector. 
For this one-dimensional constraint, an equivalent to the Wald F test is to use the test statistic $Z = \bm{c}'\bs{\hat\beta} / \sqrt{\bm{c}'\bm{V}^{CR}\bm{c}}$, which follows a standard normal distribution in large samples. 
In small samples, it is common to instead approximate the distribution of $Z$ by a $t(m - 1)$ distribution. 
\citet{Hansen2007asymptotic} provided one justification for the use of this reference distribution by identifying conditions under which $Z$ converges in distribution to $t(m-1)$ as the within-cluster sample sizes grow large, with $m$ fixed \citep[see also][]{Donald2007inference}. 
\citet{Ibragimov2010tstatistic} proposed a weighting technique derived so that that $t(m-1)$ critical values would be conservative (leading to rejection rates less than or equal to $\alpha$).
However, both of these arguments require that $\bm{c}'\bs\beta$ be separately identified within each cluster. 
Outside of these circumstances, using $t(m-1)$ critical values can still lead to over-rejection \citep{Cameron2015practitioners}. 
Furthermore, using these critical values does not take into account that the distribution of $\bm{V}^{CR}$ is affected by the structure of the covariate matrix. 
An alternative, proposed by \citet{Bell2002bias}, is to approximate the distribution of $Z$ by a $t$ distribution with degrees of freedom determined by a Satterthwaite approximation, under the working covariance model.

The t-test developed by \citet{Bell2002bias} involves using a $t(\nu)$ references distribution with degrees of freedom $\nu$, which are estimated by a Satterthwaite approximation.
The Satterthwaite approximation \citep{Satterthwaite1946approximate} entails using degrees of freedom that are a function of the the first two moments of the sampling distribution of $\bm{c}' \bm{V}^{CR} \bm{c}$.
Theoretically, these degrees of freedom should be 
\begin{equation}
\label{eq:nu_Satterthwaite}
\nu = \frac{2\left[\E\left(\bm{c}'\bm{V}^{CR2}\bm{c}\right)\right]^2}{\Var\left(\bm{c}'\bm{V}^{CR2}\bm{c}\right)}.
\end{equation}
Expressions for the first two moments of $\bm{c}'\bm{V}^{CR2}\bm{c}$ can be derived under the assumption that the errors $\bs\epsilon_1,...,\bs\epsilon_m$ are normally distributed; see Appendix \ref{app:VCR_dist}. 

In practice, both moments involve the variance structure $\bs\Sigma$, which is unknown. 
\citet{Bell2002bias} proposed to estimate the moments based on the same working model that is used to derive the adjustment matrices. 
This ``model-based'' estimate of the degrees of freedom is then calculated as 
\begin{equation}
\label{eq:nu_model}
\nu_{M} = \frac{\left(\sum_{j=1}^m \bm{p}_j' \hat{\bs\Phi} \bm{p}_j\right)^2}{\sum_{i=1}^m \sum_{j=1}^m \left(\bm{p}_i' \hat{\bs\Phi} \bm{p}_j\right)^2},
\end{equation}
where $\bm{p}_j = \left(\bm{I} - \bm{H_X}\right)_j'\bm{A}_j \bm{W}_j\bm{\ddot{R}}_j\bm{M_{\ddot{R}}} \bm{c}$.\todo{Can we use $\bm{H_{\ddot{U}}}$ here instead?} 
Alternately, for any of the CRVEs one could instead use an ``empirical'' estimate of the degrees of freedom, constructed by substituting $\bm{e}_j \bm{e}_j'$ in place of $\bs\Sigma_j$. 
However, \citet{Bell2002bias} found using simulation that this plug-in degrees of freedom estimate led to very conservative rejection rates. 

The \citet{Bell2002bias} approach has been shown to perform well in a variety of conditions (CITE simulation studies). 
These studies encompass a variety of data generation processes and covariate types. 
A key finding is that the degrees of freedom depend not only on the number of clusters $m$, but also on features of the covariates. 
When the covariate is balanced across clusters---as occurs in balanced panels with a dichotomous covariate with the same proportion of ones in each cluster---the degrees of freedom are $m - 1$ even in small samples. 
However, when the covariate exhibits large imbalances---as occurs when the panel is not balanced or if the proportion of ones varies from cluster to cluster---the degrees of freedom can be considerably smaller. 
Similarly, covariates with large leverage points will tend to exhibit lower of degrees of freedom. 
Because the degrees of freedom are covariate-dependent, it is not possible to assess whether a small-sample correction is needed based solely on the total number of clusters in the data. 
Consequently, we recommend that t-tests based on CRVE should routinely use the CR2 variance estimator and the Satterthwaite degrees of freedom, regardless even when $m$ appears to be large.

% In addition to the Satterthwaite approximation, two other promising methods for testing single-constraint hypotheses have been proposed. \citet{McCaffrey2006improved} described a saddlepoint approximation to the distribution of $Z$, providing simulation evidence that a test based on a this approximation offers even more accurate rejection rates than the Satterthwaite approximation. 
% A further method is to use a bootstrap re-sampling technique that provides small-sample refinements in the test rejection rates.
% \citet{Cameron2008bootstrap} studied several such bootstrapping techniques. 
% \citet{Webb2013wild} describe a wild boostrap procedure that performs well even when $m$ is quite small and when clusters are of unequal size.  

\subsection{Small-sample corrections for F-tests}

Little research has considered small-sample corrections for multiple-constraint hypothesis tests based on cluster-robust Wald statistics. 
We now propose an approach that approximates the distribution of $Q / q$ by a multiple of an F distribution with degrees of freedom estimated based on the working variance model. 
The sampling distribution of $Q$ is then approximated by Hotelling's $T^2$ distribution, a multiple of an $F$ distribution. 
Specifically, suppose that $\eta \bm{C}\bm{V}^{CR2} \bm{C}'$ approximately follows a Wishart distribution with $\eta$ degrees of freedom and scale matrix $\bm{C} \Var\left(\bs{\hat\beta}\right)\bm{C}'$, then 
\begin{equation}
\label{eq:AHT}
\left(\frac{\eta - q + 1}{\eta q}\right) Q \ \dot\sim \ F(q, \eta - q + 1).
\end{equation}
We will refer to this as the approximate Hotelling's $T^2$ (AHT) test.
We consider how to estimate $\eta$ below.
This approach is conceptually similar to the Satterthwaite approximation for one-dimensional constraints and reduces to it if $q = 1$. 
For $q > 1$, however, it entails a further step in the approximation\todo{What am I trying to say here?} that arise from approximating the multivariate distribution of $\bm{V}^{CR2}$, rather than the univariate distribution of a single component. 

Although not previously considered for Wald tests based on CRVE, the strategy of approximating the distribution of a robust variance estimator by a Wishart has been studied for some simpler models that are to special cases of CRVE. \citet{Zhang2013tests, Zhang2012twowayANOVA} described an AHT test for contrasts in analysis of variance models with unequal within-cell variance, which are particularly simple cases of linear models with heteroskedastic error terms. \citet{Zhang2012MANOVA} extended the method to multivariate analysis of variance models where the covariance of the errors differs across cells, a special case of model (\ref{eq:fixed_effects}) in which the CR2 variance estimator has a particularly simple form. In all of these cases, Zhang demonstrated that the robust variance estimator is a mixture of Wishart distributions that is well-approximated by a Wishart distribution with estimated degrees of freedom. Additionally, \citet{Pan2002small} described an F-test based on CR0 for use in GEE models, which also uses the Wishart approximation to the distribution of $\bm{V}^{CR}$ but estimates the degrees of freedom using a different method than the one we describe below.

The contribution of the present paper is to extend the AHT test to the more general setting of linear models with fixed effects. The remaining question is how to estimate the parameter $\eta$, which determines scalar multiplier and demoninator degrees of freedom of the F-test. 
To do so, we estimate the degrees of freedom of the Wishart distribution so that they match the mean and variance of $\bm{C}\bm{V}^{CR} \bm{C}'$ under the working variance model $\bs\Phi$, just as in the degrees of freedom for the t-test. 
The problem that arises in doing so is that when $q > 1$ it is not possible to exactly match both moments. 
\cite{Pan2002small} proposed to choose $\eta$ to minimize the squared differences between the covariances among the entries of $\eta \bm{C}\bm{V}^{CR}\bm{C}'$ and the covariances of the Wishart distribution with $\eta$ degrees of freedom and scale matrix $\bm{C}\bm{V}^{CR}\bm{C}'$. \citet{Zhang2012MANOVA} instead matches the mean and total variance of $\bm{C}\bm{V}^{CR}\bm{C}'$ (i.e., the sum of the variances of its entries), which avoids the need to calculate any covariances. In what follows we focus on this latter approach, which we have found to perform better in practice \citep{Tipton2015small}.

Let $\bm{c}_1,...,\bm{c}_q$ denote the $p \times 1$ row-vectors of $\bm{C}$. 
Let $\bm{p}_{sh} = \left(\bm{I} - \bm{H}\right)_h'\bm{A}_h'\bm{W}_h\bm{X}_h\bm{M}\bm{c}_s$ for $s = 1,...,q$ and $h = 1,...,m$. 
The degrees of freedom are then estimated under the working model as
\begin{equation}
\label{eq:eta_model}
\eta_M = \frac{\sum_{s,t=1}^q \sum_{h,i=1}^m b_{st} \bm{p}_{sh}'\hat{\bs\Phi}\bm{p}_{th} \bm{p}_{si}'\hat{\bs\Phi}\bm{p}_{ti}}{\sum_{s,t=1}^q \sum_{h,i=1}^m \bm{p}_{sh}'\hat{\bs\Phi}\bm{p}_{ti} \bm{p}_{sh}'\hat{\bs\Phi}\bm{p}_{ti} + \bm{p}_{sh}'\hat{\bs\Phi}\bm{p}_{si} \bm{p}_{th}'\hat{\bs\Phi}\bm{p}_{ti}},
\end{equation}
where $b_{st} = 1 + (s=t)$ for $s,t=1,..,q$.
Note that $\eta_M$ reduces to $\nu_M$ from Equation (\ref{eq:nu_model}) if $q = 1$.

This F-test shares features with the t-test developed by Bell and McCaffrey. Like the t-test, the degrees of freedom of this F-test depend non only on the number of clusters, but also on features of the covariates being tested. Again, these degrees of freedom can be much smaller than $m - 1$, and are particularly smaller when the covariates being tested exhibit high imbalances or leverage. Unlike the t-test case, however, in multi-parameter case, it is often more difficult to diagnose the cause of these small degrees of freedom. In some situations, however, these are straightforward extensions to the findings in t-tests. For example, if the goal is to test if there are differences across a four-arm treatment study, the degrees of freedom are largest (and close to $m - 1$) when the treatment is allocated equally across the four groups within each cluster.\todo{As in a block-randomized experiment, with all four treatment conditions in each block?} When the proportion varies across clusters, these degrees of freedom fall, often leading to degrees of freedom in the ``small sample'' territory even when the number of clusters is large. In the next section, we will illustrate these principles in several examples.

\section{EXAMPLES}
\label{subsec:examples_F}

%In each example, show a t-test and an F-test.

In this section we examine three short examples of the use of CRVE with small samples, spanning a variety of applied contexts. In the first example, the effects of substantive interest are identified within each cluster. In the second example, the effects involve between-cluster contrasts. The third example involves a cluster-robust Hausmann test for differences between within- and across-cluster information. In each example, we illustrate how the proposed small-sample t- and F-tests can be used and how they can differ from both the standard CR1 and Wild bootstrap tests. R code and data files are available for each analysis as an online supplement.

\subsubsection{Tennessee STAR class-size experiment.} 

The Tennessee STAR class size experiment is one of the most well studied interventions in education.  In the experiment, K – 3 students and teachers were randomized within each of 79 schools to one of three conditions: small class-size (targetted to have 13-17 students), regular class-size, or regular class-size with an aide (see Schazenbach, 2006 for a review). Analyses of the original study and follow up waves have found that being in a small class improves a variety of outcomes, including higher test scores \citep{Schanzenbach2006what}, increased likelihood of taking college entrance exams \citep{Krueger2001effect}, and increased rates of home ownership and earnings \citep{Chetty2011how}. 

The class-size experiment consists of three treatment conditions and multiple, student-level outcomes of possible interest. The analytic model is 
\begin{equation}
Y_{ijk} = \bm{z}_{jk}'\bs\alpha_i + \bm{x}_{jk}'\bs\beta + \gamma_k + \epsilon_{ijk}
\end{equation}
For outcome $i$, student $j$ is found in school $k$; $\bm{z}_{jk}$ includes dummies for the small-class and regular-plus-aide conditions; and the vector $\bm{x}_{jk}$ includes a set of student demographics (i.e., free or reduced lunch status; race; gender; age). Following Krueger (1999), we put the the reading, word recognition, and math scores on comparable scales by converting each outcome to percentile rankings based upon their distributions in the control condition.

We estimated the model in two ways. First, we estimated $\bs\alpha_i$ separately for each outcome $i$ and tested the null hypothesis that $\bs\alpha_i = \bm{0}$. Second, we use the seemingly unrelated regression (SUR) framework to test for treatment effects across conditions, using a simultaneous test across outcomes. In the SUR model, separate treatment effects are estimated for each outcome, but the student demographic effects and school fixed effects are pooled across outcomes. An overall test of the differences between conditions thus amounts to testing the null hypothesis that $\bs\alpha_1 = \bs\alpha_2 = \bs\alpha_3 = \bm{0}$. In all models, we estimated $\bs\alpha_i$ and $\bs\beta$ after absorbing the school fixed effects and clustered the errors by school.

\subsubsection{Heterogeneous treatment impacts} 

\citet{Angrist2009effects} reported results from a randomized trial in Israel aimed at increasing matriculation certification for post-secondary education among low achievers. 
In the Achievement Awards demonstration, 40 non-vocational high schools with the lowest 1999 certification rates nationally were selected (but with a  minimum threshold of 3\%). This included 10 Arab and 10 Jewish religious schools and 20 Jewish secular schools. The 40 schools were then pair-matched based on the 1999 certification rates, and within each pair one school was randomized to receive a cash-transfer program. In these treatment schools, every student who completed certification was eligible for a payment. The total amount at stake for a student who passed all the milestones was just under \$2,400.   

Baseline data was collected in January 2001 with follow up data collected in June 2001 and 2002. Following \citet{Angrist2009effects}, we focus on the number of certification tests taken as the outcome and report results separately for girls, for boys, and for the combined sample. Given that the program took place in three different types of schools, in this example we focus on determining if there is evidence of variation in treatment impacts across types of schools (i.e., Jewish secular, Jewish religious, and Arab). We use the analytic model:
\begin{equation}
Y_{ij} = \bm{z}_j'\bs\alpha + T_j \bm{z}_j \bs\delta + \bm{x}_{ij}'\bs\beta + \epsilon_{ij}
\end{equation}
In this model for student $i$ in school $j$, $\bm{z}_j$ is a vector of dummies indicating school type; $T_j$ is a treatment dummy indicating if school $j$ was assigned to the treatment condition; and $\bm{x}_{ij}$ contains individual student demographics (i.e., mother’s and father’s education; immigration status; number of siblings; and an indicator for the quartile of their pre-test achievement from previous years). The components of $\bs\delta$ represent the average treatment impacts in Jewish secular, Jewish religious, and Arab schools. We test the null hypothesis that $\delta_1 = \delta_2 = \delta_3$ to determine if the treatment impact differs across school types. In the second panel of Table 1 we provide the results of this test separately for boys and girls and by year. Importantly, note that the 2000 results are baseline tests, while the 2001 and 2002 results measure the effectiveness of the program.\todo{Add note about program being discontinued in 2002} 

\subsubsection{Robust Hausmann test} 

In this final example, we shift focus from analyses of experiments to panel data. Here we build off of an example first developed in \citet{Bertrand2004how} using Current Population Survey (CPS) data to relate demographics to earnings. Following \citet{Cameron2015practitioners}, we aggregated the data from the individual level to the time period, producing a balanced panel with 36 time points within 51 states (including the District of Columbia). We focus on the model,
\begin{equation}
Y_{tj} = \bm{r}_{tj}'\bs\alpha + \gamma_j + \epsilon_{ij}.
\end{equation}
In this model, time-point $t$ is nested within state $j$; the outcome $Y_{tj}$ is log-earnings, which are reported in 1999 dollars; $\bm{r}_{tj}$ includes a vector of demographic covariates specific to the time point (i.e., dummy variables for female and white; age and age-squared); and $\gamma_j$ is a fixed effect for state $j$. 

For sake of example, we focus here on determining whether to use a fixed effects (FE) estimator or a random effects (RE) estimator the four parameters in $\bs\alpha$, based on a Hausmann test. In an OLS model with uncorrelated, the Hausmann test directly compares the vectors of FE and RE estimates using a chi-squared test. However, this specification fails when cluster-robust standard errors are employed, and instead an artificial-Hausman test \citep{Arellano1993on} is typically used \citep[pp. 290-291]{Wooldridge2002econometric}. This test instead amends the model to additionally include within-cluster deviations (or cluster aggregates) of the variables of interest. In our example, this becomes,
\begin{equation}
Y_{tj} = \bm{r}_{tj}'\bs\alpha + \bm{\ddot{r}}_{tj}\bs\beta + \gamma_j + \epsilon_{tj},
\end{equation}
where $\bm{\ddot{r}}_{tj}$ denotes the vector of within-cluster deviations of the covariates (i.e., $\bm{\ddot{r}}_{tj} = \bm{r}_{tj} - \frac{1}{T}\sum_{t=1}^T \bm{r}_{tj}$).
The four parameters in $\bs\beta$ represent the differences between the within-panel and between-panel estimates of $\bs\alpha$. The artificial Hausmann test therefore reduces to testing the null hypothesis that $\bs\beta = \bm{0}$ using an F test with $q = 4$. We estimate the model using WLS with weights derived under the assumption that  $\gamma_1,...,\gamma_J$ are mutually independent, normally distributed, and independent of $\epsilon_{tj}$.

\section{SIMULATION EVIDENCE}
\label{subsec:simulations}

\section{DISCUSSION}
\label{sec:discussion}

%Current take-home points:
%\begin{itemize}
%\item How much does the working matrix matter?
%\item Role of degrees of freedom
%\item Logical consistency: t-test and F-test; FE and absorbtion
%\item Comparing these to CR1
%\item Comparing these to Wild bootstrap


%Future research:
%\begin{itemize}
%\item ``empirical'' degrees of freedom estimation
%\item use of other CR estimators
%\item computational issues with CR2 (especially when $n_j$'s are large)
%\item saddlepoint methods for $q > 1$
%\end{itemize}

\appendix

\section{BRL adjustment matrices}
\label{app:theorems}

This appendix states and provides proof of two theorems regarding the BRL adjustment matrices. 

\begin{thm}
\label{thm:BRL_FE}
Let $\bm{L} = \left(\bm{\ddot{U}}'\bm{\ddot{U}} - \bm{\ddot{U}}_j'\bm{\ddot{U}}_j\right)$ and assume that $\bm{L}$ has full rank $r + s$, so that its inverse exists. Then the adjustment matrices $\bm{A}_j$ defined in (\ref{eq:CR2_adjustment}) and (\ref{eq:CR2_Bmatrix}) satisfy criterion (\ref{eq:CR2_criterion}) and $\bm{V}^{CR2}$ is exactly unbiased when the working covariance model $\bs\Phi$ is correctly specified.
\end{thm}

\begin{proof}
The Moore-Penrose inverse of $\bm{B}_j$ can be computed from its eigen-decomposition. Let $b \leq n_j$ denote the rank of $\bm{B}_j$. 
Let $\bs\Lambda$ be the $b \times b$ diagonal matrix of the positive eigenvalues of $\bm{B}_j$ and $\bm{V}$ be the $n_j \times b$ matrix of corresponding eigen-vectors, so that $\bm{B}_j = \bm{V}\bs\Lambda\bm{V}'$. 
Then $\bm{B}_j^+ = \bm{V}\bs\Lambda^{-1}\bm{V}'$ and $\bm{B}_j^{+1/2} = \bm{V}\bs\Lambda^{-1/2}\bm{V}'$.

Now, observe that $\left(\bm{I} - \bm{H_{\ddot{R}}}\right)_j \left(\bm{I} - \bm{H_{\ddot{S}}}\right) \left(\bm{I} - \bm{H_T}\right) = \left(\bm{I} - \bm{H_X}\right)_j$. Thus, 
\begin{align}
\label{eq:step1}
\bm{\ddot{R}}_j' \bm{W}_j \bm{A}_j \left(\bm{I} - \bm{H_X}\right)_j \bs\Phi \left(\bm{I} - \bm{H_X}\right)_j' \bm{A}_j' \bm{W}_j \bm{\ddot{R}}_j &= \bm{\ddot{R}}_j' \bm{W}_j \bm{D}_j \bm{B}_j^{+1/2} \bm{B}_j \bm{B}_j^{+1/2} \bm{D}_j' \bm{W}_j \bm{\ddot{R}}_j \nonumber \\
&= \bm{\ddot{R}}_j' \bm{W}_j \bm{D}_j \bm{V}\bm{V}' \bm{D}_j' \bm{W}_j \bm{\ddot{R}}_j. 
\end{align}

Because $\bm{D}_j$, and $\bs\Phi$ are positive definite and $\bm{B}_j$ is symmetric, the eigenvectors $\bm{V}$ define an orthonormal basis for the column span of $\left(\bm{I} - \bm{H_{\ddot{U}}}\right)_j$.\todo{CITATION?} 
We now show that $\bm{\ddot{U}}_j$ is in the column space of $\left(\bm{I} - \bm{H_X}\right)_j$. 
Let $\bm{Z}_j$ be an $n_j \times (r + s)$ matrix of zeros. 
Let $\bm{Z}_k = - \bm{\ddot{U}}_k \bm{L}^{-1}\bm{M}_{\bm{\ddot{U}}}^{-1}$, for $k \neq j$ and take $\bm{Z} = \left(\bm{Z}_1',...,\bm{Z}_m'\right)'$. 
Now observe that $\left(\bm{I} - \bm{H_T}\right) \bm{Z} = \bm{Z}$. 
It follows that 
\begin{align*}
\left(\bm{I} - \bm{H_X}\right)_j \bm{Z} &= \left(\bm{I} - \bm{H_{\ddot{U}}}\right)_j \left(\bm{I} - \bm{H_T}\right) \bm{Z} = \left(\bm{I} - \bm{H_{\ddot{U}}}\right)_j \bm{Z} \\
&= \bm{Z}_j - \bm{\ddot{U}}_j\bm{M_{\ddot{U}}}\sum_{k=1}^m \bm{\ddot{U}}_k'\bm{W}_k\bm{Z}_k = \bm{\ddot{U}}_j\bm{M_{\ddot{U}}} \left(\sum_{k \neq j} \bm{\ddot{U}}_k' \bm{W}_k \bm{\ddot{U}} \right) \bm{L}^{-1}\bm{M}_{\bm{\ddot{U}}}^{-1} \\
&= \bm{\ddot{U}}_j.
\end{align*}
Thus, there exists an $N \times (r + s)$ matrix $\bm{Z}$ such that $\left(\bm{I} - \bm{H_{\ddot{U}}}\right)_j \bm{Z} = \bm{\ddot{U}}_j$, i.e., $\bm{\ddot{U}}_j$ is in the column span of $\left(\bm{I} - \bm{H_X}\right)_j$. Because $\bm{D}_j \bm{W}_j$ is positive definite and $\bm{\ddot{R}}_j$ is a sub-matrix of $\bm{\ddot{U}}_j$, $\bm{D}_j\bm{W}_j\bm{\ddot{R}}_j$ is also in the column span of $\left(\bm{I} - \bm{H_X}\right)_j$. It follows that 
\begin{equation}
\label{eq:step2}
\bm{\ddot{R}}_j' \bm{W}_j \bm{D}_j \bm{V}\bm{V}' \bm{D}_j' \bm{W}_j \bm{\ddot{R}}_j = \bm{\ddot{R}}_j' \bm{W}_j \bs\Phi_j \bm{W}_j \bm{\ddot{R}}_j.
\end{equation}
Substituting (\ref{eq:step2}) into (\ref{eq:step1}) demonstrates that $\bm{A}_j$ satisfies criterion (\ref{eq:CR2_criterion}).

Under the working model, the residuals from cluster $j$ have mean $\bm{0}$ and variance \[
\Var\left(\bm{\ddot{e}}_j\right) = \left(\bm{I} - \bm{H_X}\right)_j \bs\Phi \left(\bm{I} - \bm{H_X}\right)_j',\] 
It follows that 
\begin{align*}
\E\left(\bm{V}^{CR2}\right) &= \bm{M_{\ddot{R}}}\left[\sum_{j=1}^m \bm{\ddot{R}}_j' \bm{W}_j \bm{A}_j \left(\bm{I} - \bm{H_X}\right)_j \bs\Phi \left(\bm{I} - \bm{H_X}\right)_j' \bm{A}_j \bm{W}_j \bm{\ddot{R}}_j \right] \bm{M_{\ddot{R}}} \\
&= \bm{M_{\ddot{R}}}\left[\sum_{j=1}^m \bm{\ddot{R}}_j' \bm{W}_j \bs\Phi_j \bm{W}_j \bm{\ddot{R}}_j \right] \bm{M_{\ddot{R}}} \\
&= \Var\left(\bs{\hat\beta}\right)
\end{align*}
\end{proof}

\begin{thm}
\label{thm:absorb}
Let $\bm{\tilde{A}}_j = \bm{D}_j'\bm{\tilde{B}}_j^{+1/2} \bm{D}_j$, where $\bm{\tilde{B}}_j$ is given in (\ref{eq:CR2_B_tilde}). If $\bm{T}_j \bm{T}_k' = \bm{0}$ for $j \neq k$ and $\bm{W} = \bs\Phi^{-1}$, then $\bm{A}_j = \bm{\tilde{A}}_j$. 
\end{thm}

\begin{proof}
From the fact that $\bm{\ddot{U}}_j'\bm{W}_j\bm{T}_j = \bm{0}$ for $j = 1,...,m$, it follows that \begin{align*}
\bm{B}_j &= \bm{D}_j \left(\bm{I} - \bm{H_{\ddot{U}}}\right)_j \left(\bm{I} - \bm{H_T}\right) \hat{\bs\Phi} \left(\bm{I} - \bm{H_T}\right)' \left(\bm{I} - \bm{H_{\ddot{U}}}\right)_j' \bm{D}_j'\\
&= \bm{D}_j \left(\bm{I} - \bm{H_{\ddot{U}}} - \bm{H_T}\right)_j \hat{\bs\Phi} \left(\bm{I} - \bm{H_{\ddot{U}}} - \bm{H_T}\right)_j' \bm{D}_j' \\
&= \bm{D}_j \left(\bs\Phi_j - \bm{\ddot{U}}_j \bm{M_{\ddot{U}}}\bm{\ddot{U}}_j' - \bm{T}_j \bm{M_T}\bm{T}_j'\right)\bm{D}_j'
\end{align*}
and 
\begin{equation}
\label{eq:B_j_inverse}
\bm{B}_j^+ = \left(\bm{D}_j'\right)^{-1} \left(\bs\Phi_j - \bm{\ddot{U}}_j \bm{M_{\ddot{U}}}\bm{\ddot{U}}_j' - \bm{T}_j \bm{M_T}\bm{T}_j'\right)^+ \bm{D}_j^{-1}.
\end{equation}
Let $\bs\Omega_j = \left(\bs\Phi_j - \bm{\ddot{U}}_j \bm{M_{\ddot{U}}}\bm{\ddot{U}}_j'\right)^+$.
Using a generalized Woodbury identity \citep{Henderson1981on}, \[
\bs\Omega_j = \bm{W}_j + \bm{W}_j \bm{\ddot{U}}_j \bm{M_{\ddot{U}}}\left(\bm{M_{\ddot{U}}} - \bm{M_{\ddot{U}}} \bm{\ddot{U}}_j' \bm{W}_j \bm{\ddot{U}}_j \bm{M_{\ddot{U}}}\right)^+ \bm{M_{\ddot{U}}}\bm{\ddot{U}}_j'\bm{W}_j. \]
It follows that $\bs\Omega_j \bm{T}_j = \bm{W}_j \bm{T}_j$. 
Another application of the generalized Woodbury identity gives 
\begin{align*}
\left(\bs\Phi_j - \bm{\ddot{U}}_j \bm{M_{\ddot{U}}}\bm{\ddot{U}}_j' - \bm{T}_j \bm{M_T}\bm{T}_j'\right)^+ &= \bs\Omega_j + \bs\Omega_j \bm{T}_j \bm{M_T}\left(\bm{M_T} - \bm{M_T}\bm{T}_j' \bs\Omega_j \bm{T}_j \bm{M_T}\right)^+ \bm{M_T} \bm{T}_j' \bs\Omega_j \\
&= \bs\Omega_j + \bm{W}_j \bm{T}_j \bm{M_T}\left(\bm{M_T} - \bm{M_T}\bm{T}_j' \bm{W}_j \bm{T}_j\bm{M_T}\right)^+ \bm{M_T} \bm{T}_j' \bm{W}_j \\
&= \bs\Omega_j.
\end{align*}
The last equality follows from the fact that $\bm{T}_j \bm{M_T}\left(\bm{M_T} - \bm{M_T}\bm{T}_j' \bm{W}_j \bm{T}_j\bm{M_T}\right)^{-} \bm{M_T} \bm{T}_j' = \bm{0}$ because the fixed effects are nested within clusters. 
Substituting into (\ref{eq:B_j_inverse}), we then have that $\bm{B}_j^+ = \left(\bm{D}_j'\right)^{-1} \bs\Omega_j \bm{D}_j^{-1}$. 
But \[
\bm{\tilde{B}}_j = \bm{D}_j \left(\bm{I} - \bm{H_{\ddot{U}}}\right)_j \bs\Phi \left(\bm{I} - \bm{H_{\ddot{U}}}\right)_j' \bm{D}_j' = \bm{D}_j \left(\bs\Phi_j - \bm{\ddot{U}}_j\bm{M_{\ddot{U}}} \bm{\ddot{U}}_j'\right) \bm{D}_j' = \bm{D}_j \bs\Omega_j^+ \bm{D}_j',
\]
and so $\bm{B}_j^+ = \bm{\tilde{B}}_j^+$. It follows that $\bm{A}_j = \bm{\tilde{A}}_j$ for $j = 1,...,m$. 
\end{proof}

\section{DISTRIBUTION THEORY FOR $\bm{V}^{CR}$}
\label{app:VCR_dist}

The small-sample approximations for t-tests and F-tests both involve the distribution of the entries of $\bm{V}^{CR2}$. This appendix explains the relevant distribution theory.

First, note that any of the CR estimatosr can be written in the form $\bm{V}^{CR2} = \sum_{j=1}^M \bm{P}_j \bm{e}_j \bm{e}_j' \bm{P}_j'$ for $r \times n_j$ matrices $\bm{P}_j = \bm{M_{\ddot{R}}} \bm{\ddot{R}}_j' \bm{W}_j \bm{A}_j$.
Let $\bm{c}_1,\bm{c}_2,\bm{c}_3,\bm{c}_4$ be fixed, $p \times 1$ vectors and consider the linear combination $\bm{c}_1' \bm{V}^{CR2} \bm{c}_2$. 
\citet[Theorem 4]{Bell2002bias} show that the linear combination is a quadratic form in $\bm{y}$: \[
\bm{c}_1' \bm{V}^{CR2} \bm{c}_2 = \bm{y}'\left(\sum_{j=1}^m \bm{p}_{2j} \bm{p}_{1j}'\right) \bm{y}, \]
for $N \times 1$ vectors $\bm{p}_{sh} = \left(\bm{I} - \bm{H_X}\right)_h' \bm{P}_h' \bm{c}_s$, $s = 1,...,4$, and $h = 1,...,m$. 

Standard results regarding quadratic forms can be used to derive the moments of the linear combination \citep[e.g.,][Sec. 13.5]{Searle2006matrix}. We now assume that $\bs\epsilon_1,...,\bs\epsilon_m$ are multivariate normal with zero mean and variance $\bs\Sigma$. It follows that 
\begin{align}
\label{eq:CRVE_expectation}
\E\left(\bm{c}_1' \bm{V}^{CR2} \bm{c}_2\right) &= \sum_{j=1}^m \bm{p}_{1j}' \bs\Sigma \bm{p}_{2j} \\
\label{eq:CRVE_variance}
\Var\left(\bm{c}_1' \bm{V}^{CR2} \bm{c}_2\right) &= \sum_{i=1}^m \sum_{j=1}^m \left(\bm{p}_{1i}' \bs\Sigma \bm{p}_{2j}\right)^2 + \bm{p}_{1i}' \bs\Sigma \bm{p}_{1j} \bm{p}_{2i}' \bs\Sigma \bm{p}_{2j} \\
\label{eq:CRVE_covariance}
\Cov\left(\bm{c}_1' \bm{V}^{CR2} \bm{c}_2, \bm{c}_3' \bm{V}^{CR} \bm{c}_4\right) &= \sum_{i=1}^m \sum_{j=1}^m \bm{p}_{1i}' \bs\Sigma \bm{p}_{4j} \bm{p}_{2i}' \bs\Sigma \bm{p}_{3j} + \bm{p}_{1i}' \bs\Sigma \bm{p}_{3j} \bm{p}_{2i}' \bs\Sigma \bm{p}_{4j}.
\end{align}
Furthermore, the distribution of $\bm{c}_1' \bm{V}^{CR2} \bm{c}_2$ can be expressed as a weighted sum of $\chi^2_1$ distributions \citep{mathai1992quadratic}, with weights given by the eigen-values of the $m \times m$ matrix with $\left(i,j\right)^{th}$ entry $\bm{p}_{1i}' \bs\Sigma \bm{p}_{2j}$, $i,j=1,...,m$.

\bibliographystyle{agsm}
\bibliography{bibliography}

\end{document}

