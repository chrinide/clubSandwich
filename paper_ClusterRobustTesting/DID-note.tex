\documentclass{article}\usepackage[]{graphicx}\usepackage[]{color}
%% maxwidth is the original width if it is less than linewidth
%% otherwise use linewidth (to make sure the graphics do not exceed the margin)
\makeatletter
\def\maxwidth{ %
  \ifdim\Gin@nat@width>\linewidth
    \linewidth
  \else
    \Gin@nat@width
  \fi
}
\makeatother

\definecolor{fgcolor}{rgb}{0.345, 0.345, 0.345}
\newcommand{\hlnum}[1]{\textcolor[rgb]{0.686,0.059,0.569}{#1}}%
\newcommand{\hlstr}[1]{\textcolor[rgb]{0.192,0.494,0.8}{#1}}%
\newcommand{\hlcom}[1]{\textcolor[rgb]{0.678,0.584,0.686}{\textit{#1}}}%
\newcommand{\hlopt}[1]{\textcolor[rgb]{0,0,0}{#1}}%
\newcommand{\hlstd}[1]{\textcolor[rgb]{0.345,0.345,0.345}{#1}}%
\newcommand{\hlkwa}[1]{\textcolor[rgb]{0.161,0.373,0.58}{\textbf{#1}}}%
\newcommand{\hlkwb}[1]{\textcolor[rgb]{0.69,0.353,0.396}{#1}}%
\newcommand{\hlkwc}[1]{\textcolor[rgb]{0.333,0.667,0.333}{#1}}%
\newcommand{\hlkwd}[1]{\textcolor[rgb]{0.737,0.353,0.396}{\textbf{#1}}}%

\usepackage{framed}
\makeatletter
\newenvironment{kframe}{%
 \def\at@end@of@kframe{}%
 \ifinner\ifhmode%
  \def\at@end@of@kframe{\end{minipage}}%
  \begin{minipage}{\columnwidth}%
 \fi\fi%
 \def\FrameCommand##1{\hskip\@totalleftmargin \hskip-\fboxsep
 \colorbox{shadecolor}{##1}\hskip-\fboxsep
     % There is no \\@totalrightmargin, so:
     \hskip-\linewidth \hskip-\@totalleftmargin \hskip\columnwidth}%
 \MakeFramed {\advance\hsize-\width
   \@totalleftmargin\z@ \linewidth\hsize
   \@setminipage}}%
 {\par\unskip\endMakeFramed%
 \at@end@of@kframe}
\makeatother

\definecolor{shadecolor}{rgb}{.97, .97, .97}
\definecolor{messagecolor}{rgb}{0, 0, 0}
\definecolor{warningcolor}{rgb}{1, 0, 1}
\definecolor{errorcolor}{rgb}{1, 0, 0}
\newenvironment{knitrout}{}{} % an empty environment to be redefined in TeX

\usepackage{alltt}
\usepackage[margin=1in]{geometry}
\usepackage{amsmath}
\usepackage{amsthm}
\RequirePackage[natbibapa]{apacite}

\newcommand{\Prob}{\text{Pr}}
\newcommand{\E}{\text{E}}
\newcommand{\Cov}{\text{Cov}}
\newcommand{\corr}{\text{corr}}
\newcommand{\Var}{\text{Var}}
\newcommand{\iid}{\stackrel{\text{iid}}{\sim}}
\newcommand{\tr}{\text{tr}}
\newcommand{\bm}{\mathbf}
\newcommand{\bs}{\boldsymbol}
\IfFileExists{upquote.sty}{\usepackage{upquote}}{}
\begin{document}

\section{Basic difference-in-differences example}

Consider a simple difference-in-differences design with $m$ clusters and $n = 2$ time periods. 
Suppose that the first $m_0$ clusters remain untreated in the second time period and the remaining $m_1 = m - m_0$ clusters are treated in the second time period.
The basic difference-in-differences model for this design is then
\begin{equation}
\label{eq:DID}
y_{it} = \alpha_i + \beta_t + \delta T_{it} + e_{it},
\end{equation}
where $T_{i1} = 1$ for $i = m_0 + 1,...,m$, $T_{it} = 0$ otherwise, and $\delta$ is the average treatment effect. 

Estimating $\delta$ by OLS is exactly equivalent to taking first differences and then calculating the mean difference between treated and untreated clusters. Let $d_i = y_{i1} - y_{i0}$ for $i = 1,...,m$, $\bar{d}_0 = \sum_{i=1}^{m_0} d_i / m_0$, and $\bar{d}_1 = \sum_{i=m_0 + 1}^m d_i / m_1$. Then $\hat\delta = \bar{d}_1 - \bar{d}_0$. In this simplified representation of the model, it is clear that the null hypothesis $\delta = 0$ may be tested using a simple two-sample t-test on the difference scores, while allowing for unequal variances. The sampling variance of $\hat\delta$ can be estimated from the difference scores as 
\[
V_{\Delta} = \frac{1}{m_0 (m_0 - 1)}\sum_{i=1}^{m_0} \left(d_i - \bar{d}_0\right)^2 + \frac{1}{m_1 (m_1 - 1)}\sum_{i=m_0 + 1}^{m} \left(d_i - \bar{d}_1\right)^2.
\]
Under a "working homoskedasticity" model, the degrees of freedom corresponding to $V_{\Delta}$ are 
\[
\nu_\Delta = \frac{m^2 (m_0 - 1) (m_1 - 1)}{m_0^2 (m_0 - 1) + m_1^2 (m_1 - 1)}
\]
\citep{Imbens2015robust}.

We shall now consider the variance estimator and degrees of freedom generated by the CR2 correction as applied to the full difference-in-differences model (\ref{eq:DID}), while estimating $\delta$ after absorbing the cluster- and period-specific effects.
We use the "working independence" model for deriving the CR2 adjustment matrices and degrees of freedom.
Following the notation of the main paper, this design has
\[
\bm{R}_i = \left[\begin{array}{c} 0 \\ T_{i1}\end{array}\right] \qquad
\bm{S}_i = \left[\begin{array}{c} 0 \\ 1 \end{array}\right] \qquad
\bm{T}_i = \left[\begin{array}{c} 1 \\ 1 \end{array}\right] \left[\begin{array}{cccc}I(i=1) & I(i=2) & \cdots & I(i=m)\end{array}\right]
\]
After absorption, $\bm{\ddot{R}}_i = \left(T_{i1} - m_1 / m\right) / 2 \left[\begin{array}{c}-1 \\ 1 \end{array}\right]'$, $\bm{M_{\ddot{R}}} = 2 m / (m_0 m_1)$, and 
\[
\bm{e}_i = \frac{d_i - \bar{d}_0}{2} \left[\begin{array}{c} 1 \\ 1 \end{array}\right] \quad \text{for} \quad i = 1,...,m_0, \qquad \bm{e}_i = \frac{d_i - \bar{d}_1}{2} \left[\begin{array}{c} 1 \\ 1 \end{array}\right], \quad \text{for} \quad i = m_0 + 1,...,m.
\]
If the CR2 adjustment matrices are calculated based on the absorbed model only, then 
\[
\bm{A}_i = \left(\bm{I}_i - \bm{\ddot{R}}_i \bm{M_{\ddot{R}}} \bm{\ddot{R}}_i'\right)^{+1/2} = \left[\begin{array}{cc} 1 + a_i & - a_i \\ -a_i & 1 + a_i \end{array}\right],
\]
where
\begin{align*}
a_i &= \frac{1}{2}\left(\sqrt{\frac{m_0 m}{m_0 m - m_1}} - 1\right) \qquad i = 1,...,m_0 \\
a_i &= \frac{1}{2}\left(\sqrt{\frac{m_1 m}{m_1 m - m_0}} - 1\right) \qquad i = m_0 + 1,...,m.
\end{align*}
Using these adjustment matrices yields the variance estimator 
\[
V_{\bm{\ddot{R}}} = \frac{1}{m_0 (m_0 - m_1 / m)} \sum_{i=1}^{m_0} \left(d_i - \bar{d}_0\right)^2 + \frac{1}{m_1 (m_1 - m_0 / m)}\sum_{i=m_0 + 1}^{m} \left(d_i - \bar{d}_1\right)^2,
\]
which will be slightly smaller than $V_\Delta$, with Satterthwaite degrees of freedom
\[
\nu_{\bm{\ddot{R}}} = \frac{\displaystyle{\left(\frac{m_0 - 1}{m_0(m_0 - m_1 / m)} + \frac{m_1 - 1}{m_1(m_1 - m_0 / m)}\right)^2}}{\displaystyle{\frac{1}{m_0(m_0 - m_1 / m)} + \frac{1}{m_1(m_1 - m_0 / m)}}},
\]
which will be slightly larger than $\nu_{\Delta}$.

Now consider calculating the adjustment matrices using the full design matrix, as recommended in the paper. Theorem 2 implies that the adjustment matrices can be calculated from $\bm{\ddot{U}}$, ignoring the cluster-specific effects. We then have
\[
\bm{A}_i = \left(\bm{I}_i - \bm{\ddot{U}}_i \bm{M_{\ddot{U}}} \bm{\ddot{U}}_i'\right)^{+1/2} = \left[\begin{array}{cc} 1 + b_i & - b_i \\ -b_i & 1 + b_i \end{array}\right],
\]
where
\begin{align*}
b_i &= \frac{1}{2}\left(\sqrt{\frac{m_0}{m_0 - 1}} - 1\right) \qquad i = 1,...,m_0 \\
b_i &= \frac{1}{2}\left(\sqrt{\frac{m_1}{m_1 - 1}} - 1\right) \qquad i = m_0 + 1,...,m.
\end{align*}
It can be verified that using these adjustment matrices yields a variance estimator that is exactly equivalent to $V_{\Delta}$, with degrees of freedom equal to $\nu_\Delta$.



\end{document}
